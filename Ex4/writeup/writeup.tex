% class
\documentclass[ngerman]{scrartcl}


% text input and font
\usepackage[utf8]{inputenc}  % input in UTF-8
\usepackage[T1]{fontenc}  % output in T1 fonts (west European encoding)
\usepackage{lmodern}  % Latin modern font
\usepackage{babel}  % language package
\usepackage{amsmath, amssymb, amstext, mathtools}  % math packages (American Math Society) + correction of amsmath (mathtools) [loads amsmath automatically]
\usepackage{physics}  % macros for easier typesetting of physical formulas
\usepackage{letltxmacro}  % \let command for robust macros (new sqrt)


% page geometry
\usepackage{scrlayer-scrpage}  % page formatting with KOMA options
\usepackage[paper=a4paper, hmargin=3cm, vmargin=2.5cm, includehead, includefoot]{geometry}


% floats
\usepackage[hypcap=false, labelfont=bf]{caption, subcaption}  % caption editing - hypcap warning with hyperref
\usepackage{float}  % for [H] (forced here) specifier
\usepackage{tabularray}
\usepackage{diagbox}  % table cells with diagonal lines


% graphical input
\usepackage{graphicx}  % input JPEG, PNG, PDF, etc.
\usepackage{pdfpages}  % input PDF as whole pages
\usepackage{lastpage}  % reference to last page


% text
\usepackage[locale=DE, uncertainty-mode=separate]{siunitx}  % SI units, German formatting - \pm stays \pm instead of ..(.)
\usepackage{icomma}  % no space after commas instead of English points) in decimal values
\usepackage{enumitem}  % better enumerating with style options
\usepackage{nicefrac}  % inline-fractions in n/d-style
\usepackage{fancyvrb}  % Verbatim environment with better options (capital V!)

% literacy
\usepackage[style=apa]{biblatex}  % backend=Biber is standard
\usepackage{csquotes}  % better quotation - should also be used in combination with package babel (warning)
\usepackage{xurl}  % breaks links - after BibLaTeX, but before hyperref!
\usepackage[hidelinks]{hyperref}  % produces most errors, last to load


% KOMA setups
% header and footer
\pagestyle{scrheadings}  % KOMA style
\clearpairofpagestyles  % reset
\setkomafont{pageheadfoot}{\normalfont}  % standard font in header and footer
\setlength{\headheight}{27.2pt}  % just look at the warning
\ihead{DA I - Halden}  % inner (left) head
\chead{\textsc{Wachmann} Elias (12004232)}  % center head
\ohead{\today}  % outer (right) head
\cfoot{\pagemark{} / \pageref*{LastPage}}  % center foot - *: ref but no hyperlink
% {}: empty statement
% \ : protected space
% \,: small space
\DeclareTOCStyleEntry{dottedtocline}{section}  % sections in TableOfContents with dotted lines
\KOMAoptions{parskip=half-}  % paragraphs with half a line height space instead of indentation, last line with no special treatment


% package setups

% BibLaTeX source
% \addbibresource{halbleiterdiode.bib}


% rewrite names (babel overwrites German with standard English names, therefore at document beginn [after everything is loaded])
\AtBeginDocument{\renewcommand{\refname}{Literaturverzeichnis}}
% others:
% \contentsname
% \listtablename
% \listfigurename


% new sqrt
% https://en.wikibooks.org/wiki/LaTeX/Mathematics
\makeatletter
\let\oldr@@t\r@@t
\def\r@@t#1#2{%
\setbox0=\hbox{$\oldr@@t#1{#2\,}$}\dimen0=\ht0
\advance\dimen0-0.2\ht0
\setbox2=\hbox{\vrule height\ht0 depth -\dimen0}%
{\box0\lower0.4pt\box2}}
\LetLtxMacro{\oldsqrt}{\sqrt}
\renewcommand*{\sqrt}[2][\ ]{\oldsqrt[#1]{#2} }
\makeatother

 
% tabularray
% imports_and_setups{
%     expl3,
%     xparse,
%     ninecolors
%     \hypersetup{pdfborder={0 0 0}}
% }
\RequirePackage{tabularray}  % like \usepackage but for packages
\UseTblrLibrary{siunitx}  % siunitx suited for tabularray
% TRIPLE BRACKETS {{{}}} to protect strings from \num{} interpretation in S columns
\UseTblrLibrary{diagbox}  % diagbox suited for tabularray
\UseTblrLibrary{varwidth}  % measure cell width using package 'varwidth'

% standard environment
\SetTblrInner{
    hlines,
    vlines,
    columns={
            halign=c,
            valign=m,
        },
    measure=vbox,
}

% X columns
\NewTblrEnviron{tblrx}
\SetTblrInner[tblrx]{
    hlines,
    vlines,
    columns={
            halign=c,
            valign=m,
            co=1,  % coefficient of width for expendable columns (X columns)
        },
    width=\linewidth,
    vspan=minimal,
    measure=vbox,
}

% -X columns
\NewTblrEnviron{tblr-x}
\SetTblrInner[tblr-x]{
    hlines,
    vlines,
    columns={
            halign=c,
            valign=m,
            co=-1,  % shrinks X column down to natural width
        },
    width=\linewidth,
    vspan=minimal,
    measure=vbox,
}

% no hline and vline left and on top of first cell
\NewTblrEnviron{tblr_omit_first_cell}
\SetTblrInner[tblr_omit_first_cell]{
    hlines,
    vlines,
    columns={
            halign=c,
            valign=m,
        },
    hspan=even,
    vspan=minimal,
    %
    hline{1}={1}{white},  % first row, only first cell
    vline{1}={1}{white},
    measure=vbox,
}

\usepackage{pythonhighlight}
\usepackage{amsmath}
\usepackage{algorithm}
\usepackage[noend]{algpseudocode}

% individual settings
%\addbibresource{beschreibung.bib}  % database import with absolute path - file ending!
\newcommand{\code}{\texttt}

\begin{document}
\begin{titlepage}
    \begin{center}
        \vspace{1cm}
        \Huge
        \textbf{Data Structures and Algorithms I}
        \vspace{5mm}

        \Large
        Homework Assignment 4 - Halden
        \vspace{5mm}


        \textbf{Wachmann Elias}


        \today
    \end{center}
\end{titlepage}

\section{Aufgabenstellung}
\label{sec:aufgabenstellung}
A \emph{d}-ary heap is like a binary heap, but nonleaf nodes have \emph{d} children instead of 2 children.
\begin{enumerate}[label=(\Alph*)]
    \item How would you represent a \emph{d}-ary heap in a array?
    \item What is the height of a \emph{d}-ary heap of $n$ elements in terms of $n$ and $d$? Justify your answer.
    \item Give an efficient implementation of \code{HEAPIFY} in a \emph{d}-ary max-heap. Analyze its running time in terms of $d$ and $n$.
\end{enumerate}

\section{A - Darstellung in einem Array}
\label{sec:darstellung}

Gleich wie bei einer binären Halde wird der Maximale wert an den Anfang des Arrays (im folgendem \code{A}), also \code{A[0]} gespeichert. In einer \emph{d}-äre Halde hat nun jeder Knoten (außer leafs) $d$ Kinder. Diese $d$ Kinder werden nun an die nächsten $d$ Stellen im eindimensionalen Array gespeichert. Damit sind die Kinder der Wurzel \code{A[0]} im Array an den Stellen \code{A[1]} bis \code{A[d]}. Die Nächste Ebene im Baum hat $d\cdot d$ Kinder, welche im Array von \code{A[d+1]} bis \code{A[$\code{d}^2$+d]}. Abermals die nächste von \code{A[$\code{d}^2$+d+1]} bis \code{A[$\code{d}^3$+$\code{d}^2$+d]} Allgemein liegt die $n$-te Ebene (Wurzel ist 0. Ebene) von \code{A[$\sum_{i = 0}^{n}$d$^i$]} bis \code{A[($\sum_{i = 0}^{n+1}$d$^i$)-1]} im 0-indiziertem Array.

\section{B - Höhe einer \emph{d}-ären Halde}
\label{sec:hoehe}
Wie schon in \autoref{sec:darstellung} ersichtlich ist, wächst eine \emph{d}-ären Halde um $h^d$ - Kinder wobei $h$ die Ebene/Höhe angibt. Die Anzahl $n$ der in einer Halde gespeicherten Elemente liegt nun sicherlich wie folgt:
\begin{align*}
    1+\sum_{i = 0}^{h-1} d^i &\leq n \leq \sum_{i = 0}^{h} d^i\\
    1+\frac{d^h-1}{d-1} &\leq n \leq \frac{d^{h+1}-1}{d-1}
\end{align*}
Nimmt man nun davon $\log_d$ und formt beide Seiten jeweils auf $h$ um erhält man:
\begin{align*}
    \log_d(n(d-1)+1)-1 &\leq h \leq \log_d((n-1)(d-1)+1)\\
    \log_d(n(d-1)+1) &\leq h+1 \rightarrow \lfloor \log_d(n(d-1)+1) \rfloor  = h \rightarrow h = \Omega(\log_d(n))\\
    \log_d((n-1)(d-1)+1) &\geq h \rightarrow \lceil \log_d((n-1)(d-1)+1) \rceil = h \rightarrow h = \mathcal{O}(\log_d(n))
\end{align*}
Daraus folgt nun schließlich, dass die Höhe sich zu
\begin{align*}
    h = \Theta(\log_d(n))
\end{align*}
ergibt.
% Da $1+\sum_{i = 0}^{h-1} d^i$ trivialerweise größer als $\sum_{i = 0}^{h-1} d^i$ ist, muss somit auch $h-1 < \log_d(n(d-1)+1)-1 < h$ gelten. Daraus folgt schließlich, dass die Höhe durch den folgenden Ausdruck gegeben ist:
% \begin{align*}
%     h = \lceil \log_d(n(d-1)+1)\rceil - 1 = \lceil \log_d(nd-n+1)\rceil - 1 \\ 
% \end{align*}
\section{C - Implementation von \code{HEAPIFY}}
\label{sec:implementation}
Es folgt eine Implementation von \code{HEAPIFY}.
\begin{algorithm}[H]
    \label{alg:heapify}
    \caption{\code{HEAPIFY}}
    \begin{algorithmic}[1]
        \Function{heapify}{A,i,d}
            \State{// Input: \code{A} array to heapify}
            \State{//        \code{i} index which should be heapified}
            \State{//        \code{d} order of \emph{d}-ary heap}
            \State{\code{n} $\gets$ length of \code{A}}
            \State{\code{kids} $\gets$ [ ]}
            \Comment{Setup empty array for kid indices}
            \For{\code{count} $\gets$ \code{1} to \code{d}}
            \Comment{including d}
            \State{\code{kids[count-1] $\gets$ d*i+count}}
            \EndFor
            \State{\code{index} $\gets$ i}
            \For{\code{count} $\gets$ \code{0} to \code{(}length of \code{kids)-1}}
            \If{\code{kid} < \code{n} and \code{A[kids[count]]} > \code{A[index]}}
            \State{\code{index} $\gets$ \code{k}}
            \EndIf
            \EndFor
            \If{\code{i} does not equal \code{index}}
            \State{switch \code{A[index]} and \code{A[i]}}
            \State{\code{HEAPIFY(A,index,d)}}
            \EndIf
        \EndFunction
\end{algorithmic}
\end{algorithm}

\subsection{Laufzeitanalyse von \code{HEAPIFY}}
\label{subsec:laufzeit}
Bla Bla Bla
\clearpage
\end{document}