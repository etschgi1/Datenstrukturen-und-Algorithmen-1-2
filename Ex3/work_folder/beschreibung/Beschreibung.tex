% class
\documentclass[ngerman]{scrartcl}


% text input and font
\usepackage[utf8]{inputenc}  % input in UTF-8
\usepackage[T1]{fontenc}  % output in T1 fonts (west European encoding)
\usepackage{lmodern}  % Latin modern font
\usepackage{babel}  % language package
\usepackage{amsmath, amssymb, amstext, mathtools}  % math packages (American Math Society) + correction of amsmath (mathtools) [loads amsmath automatically]
\usepackage{physics}  % macros for easier typesetting of physical formulas
\usepackage{letltxmacro}  % \let command for robust macros (new sqrt)


% page geometry
\usepackage{scrlayer-scrpage}  % page formatting with KOMA options
\usepackage[paper=a4paper, hmargin=3cm, vmargin=2.5cm, includehead, includefoot]{geometry}


% floats
\usepackage[hypcap=false, labelfont=bf]{caption, subcaption}  % caption editing - hypcap warning with hyperref
\usepackage{float}  % for [H] (forced here) specifier
\usepackage{tabularray}
\usepackage{diagbox}  % table cells with diagonal lines


% graphical input
\usepackage{graphicx}  % input JPEG, PNG, PDF, etc.
\usepackage{pdfpages}  % input PDF as whole pages
\usepackage{lastpage}  % reference to last page


% text
\usepackage[locale=DE, uncertainty-mode=separate]{siunitx}  % SI units, German formatting - \pm stays \pm instead of ..(.)
\usepackage{icomma}  % no space after commas instead of English points) in decimal values
\usepackage{enumitem}  % better enumerating with style options
\usepackage{nicefrac}  % inline-fractions in n/d-style
\usepackage{fancyvrb}  % Verbatim environment with better options (capital V!)

% literacy
\usepackage[style=apa]{biblatex}  % backend=Biber is standard
\usepackage{csquotes}  % better quotation - should also be used in combination with package babel (warning)
\usepackage{xurl}  % breaks links - after BibLaTeX, but before hyperref!
\usepackage[hidelinks]{hyperref}  % produces most errors, last to load


% KOMA setups
% header and footer
\pagestyle{scrheadings}  % KOMA style
\clearpairofpagestyles  % reset
\setkomafont{pageheadfoot}{\normalfont}  % standard font in header and footer
\setlength{\headheight}{27.2pt}  % just look at the warning
\ihead{DA I\\Sortieralgorithmen}  % inner (left) head
\chead{\textsc{Wachmann} Elias (12004232)}  % center head
\ohead{\today}  % outer (right) head
\cfoot{\pagemark{} / \pageref*{LastPage}}  % center foot - *: ref but no hyperlink
% {}: empty statement
% \ : protected space
% \,: small space
\DeclareTOCStyleEntry{dottedtocline}{section}  % sections in TableOfContents with dotted lines
\KOMAoptions{parskip=half-}  % paragraphs with half a line height space instead of indentation, last line with no special treatment


% package setups

% BibLaTeX source
\addbibresource{halbleiterdiode.bib}


% rewrite names (babel overwrites German with standard English names, therefore at document beginn [after everything is loaded])
\AtBeginDocument{\renewcommand{\refname}{Literaturverzeichnis}}
% others:
% \contentsname
% \listtablename
% \listfigurename


% new sqrt
% https://en.wikibooks.org/wiki/LaTeX/Mathematics
\makeatletter
\let\oldr@@t\r@@t
\def\r@@t#1#2{%
\setbox0=\hbox{$\oldr@@t#1{#2\,}$}\dimen0=\ht0
\advance\dimen0-0.2\ht0
\setbox2=\hbox{\vrule height\ht0 depth -\dimen0}%
{\box0\lower0.4pt\box2}}
\LetLtxMacro{\oldsqrt}{\sqrt}
\renewcommand*{\sqrt}[2][\ ]{\oldsqrt[#1]{#2} }
\makeatother

 
% tabularray
% imports_and_setups{
%     expl3,
%     xparse,
%     ninecolors
%     \hypersetup{pdfborder={0 0 0}}
% }
\RequirePackage{tabularray}  % like \usepackage but for packages
\UseTblrLibrary{siunitx}  % siunitx suited for tabularray
% TRIPLE BRACKETS {{{}}} to protect strings from \num{} interpretation in S columns
\UseTblrLibrary{diagbox}  % diagbox suited for tabularray
\UseTblrLibrary{varwidth}  % measure cell width using package 'varwidth'

% standard environment
\SetTblrInner{
    hlines,
    vlines,
    columns={
            halign=c,
            valign=m,
        },
    measure=vbox,
}

% X columns
\NewTblrEnviron{tblrx}
\SetTblrInner[tblrx]{
    hlines,
    vlines,
    columns={
            halign=c,
            valign=m,
            co=1,  % coefficient of width for expendable columns (X columns)
        },
    width=\linewidth,
    vspan=minimal,
    measure=vbox,
}

% -X columns
\NewTblrEnviron{tblr-x}
\SetTblrInner[tblr-x]{
    hlines,
    vlines,
    columns={
            halign=c,
            valign=m,
            co=-1,  % shrinks X column down to natural width
        },
    width=\linewidth,
    vspan=minimal,
    measure=vbox,
}

% no hline and vline left and on top of first cell
\NewTblrEnviron{tblr_omit_first_cell}
\SetTblrInner[tblr_omit_first_cell]{
    hlines,
    vlines,
    columns={
            halign=c,
            valign=m,
        },
    hspan=even,
    vspan=minimal,
    %
    hline{1}={1}{white},  % first row, only first cell
    vline{1}={1}{white},
    measure=vbox,
}

\usepackage{pythonhighlight}
\usepackage{amsmath}
\usepackage{algorithm}
\usepackage[noend]{algpseudocode}

% individual settings
%\addbibresource{beschreibung.bib}  % database import with absolute path - file ending!
\newcommand{\code}{\texttt}



\begin{document}
\begin{titlepage}
    \begin{center}
        \vspace{1cm}
        \Huge
        \textbf{Data Structures and Algorithms I}
        \vspace{5mm}

        \Large
        Homework Assignment 3
        \vspace{5mm}


        \textbf{Wachmann Elias}


        \today
    \end{center}
\end{titlepage}

\clearpage
\tableofcontents
\newpage

\section{Algotithmus in Worten \& Pseudo-Code}
\label{sec:pseudo_code}
Aufgabenstellung ist es, aus einer Liste mit beliebig vielen Produkt-Reviews eine abwärts-sortierte Liste mit der Häufigkeit von k-hintereinander-stehenden Worten \code{c} und eine zugehörige Liste dieser Worte \code{y} zu generieren. 

\textbf{Der Algorithmus in Worten}

Dem Algorithmus \code{count\_vectorizer} wird eine list of lists namens \code{texts} und die Anzahl der aufeinanderfolgenden Wörter \code{k} übergeben.
Zuerst werden für jede Review in \code{texts} immer k-hintereinander-stehende Wörter in einem String konkateniert und in ein Python \code{Dict} gespeichert, hierbei wird zuerst versucht den String als Key zu verwenden, um so den count um $1$ zu erhöhen. Schlägt dies fehl, so wird ein neuer Key mit count = $1$ angelegt. Nun werden die Keys und Values des Dictionaries in die beiden Listen \code{y} bzw. \code{c} entpackt. Nun wird \code{c} mittels \code{merge\_sort} absteigend sortiert, dabei wird die Liste der Worte \code{y} gleich sortiert, sodass weiterhin jede Stelle in \code{c} die Anzahl der dazugehörigen Phrase in \code{y} gibt.

\textbf{Pseudo-Code}

\begin{algorithm}[H]
\label{alg:merge}
\caption{\code{merge\_}}
\begin{algorithmic}[1]
    \Function{merge\_}{words, weights, start, mid, end}
    \State{// Input: \code{words} array of phrases with \code{k} consecutive words}
    \State{//        \code{weights} for merging (corresponding count of phrases, merging in desc. order)}
    \State{//        \code{start} start index for merge}
    \State{//        \code{mid} mid index for merge}
    \State{//        \code{end} end index for merge}
    \State{// Output: \code{words} array of unique phrases (locally sorted between start and end)}
    \State{//         \code{weights} array of corresponding counts of phrases in \code{words})}
    \For{\code{left} $\gets$ \code{start} to \code{mid}}
        \State{\code{left\_arr[left - start]} $\gets$ (\code{words[left]}, \code{weights[left]})}
    \EndFor
    \For{\code{right} $\gets$ \code{mid + 1} to \code{end}}
        \State{\code{right\_arr[right - mid]} $\gets$ (\code{words[right]}, \code{weights[right]})} 
    \EndFor
    \State{\code{left\_arr[mid + 1]} $\gets$ (`` '', 0)}
    \State{\code{right\_arr[end + 1]} $\gets$ (`` '', 0)}
    \State{\code{left} $\gets 0$; \code{right} $\gets 0$}
    \For{\code{counter} $\gets$ \code{start} to \code{end}}
        \If{\code{left\_arr[left][1]} $\geq$ \code{right\_arr[right][1]}}
            \State{\code{words[counter]} $\gets$ \code{left\_arr[left][0]}}
            \State{\code{weights[counter]} $\gets$ \code{left\_arr[left][1]}}
            \State{\code{left} $\gets$ \code{left} + 1}
        \Else
            \State{\code{words[counter]} $\gets$ \code{right\_arr[right][0]}}
            \State{\code{weights[counter]} $\gets$ \code{right\_arr[right][1]}}
            \State{\code{right} $\gets$ \code{right} + 1}
        \EndIf
    \EndFor
    \State{\textbf{return} \code{(words, weights)}}
    \EndFunction
\end{algorithmic}
\end{algorithm}

\begin{algorithm}[H]
\label{alg:merge_sort}
\caption{\code{merge\_sort}}
\begin{algorithmic}[1]
    \Function{merge\_sort}{words, weights, start, end}
    \State{// Input: \code{words} array of phrases with \code{k} consecutive words}
    \State{//        \code{weights} for sort (corresponding count of phrases, sorting in desc. order)}
    \State{//        \code{start} start index for merge\_sort}
    \State{//        \code{end} end index for merge\_sort}
    \State{// Output: \code{words} array of unique phrases sorted in desc. order}
    \State{//         \code{weights} array of corresponding counts of phrases in \code{words})}
        \If{\code{start} < \code{end}}
            \State{\code{mid} $\gets$ round down \code{(start + end)/2}}
            \State{\code{MERGE\_SORT(words,weights,start,mid)}}
            \State{\code{MERGE\_SORT(words,weights,mid + 1,end)}}
            \State{\textbf{return} \code{MERGE\_(words,weights,start,mid,end)}}
        \Else
            \State{\textbf{return} \code{(words, weights)}}
        \EndIf
    \EndFunction
\end{algorithmic}
\end{algorithm}


\begin{algorithm}[H]
\label{alg:vectorizer}
\caption{\code{count\_vectorizer}}
\begin{algorithmic}[1]
    \Function{count\_vectorizer}{texts, k=1}
    \State{// Input: \code{texts} array of arrays with each containing the words from a review}
    \State{//        \code{k} integer with the count of consecutive words in a phrase (defaults to 1)}
    \State{// Output: \code{y} array of unique phrases (with \code{k} consecutive words)} 
    \State{// \code{c} array of occurrence count for corresponding index in \code{y}}
    \State $\code{y} \gets \text{[ ]}$
    \State $\code{c} \gets \text{[ ]}$
    \State \code{vals} $\gets$ \{ \}
    \Comment{empty Hashmap}
    \State \code{entries} $\gets 0$
    \For{\code{text} $\gets 0$ to $\text{length of } \code{texts}$}
        \State{\code{counter} $\gets 0$}
        \State{\code{len\_} $\gets \text{length of } \code{texts}$}
        \While{\code{counter} $\leq$ (\code{len\_} - \code{k})}
            % \State{\code{word} $\gets$ ""}
            \If{k != 1}
                \State \code{word} $\gets$ concatenate \code{texts[text]} from counter to counter+1 separate with spaces
            \Else
                \State \code{word} $\gets$ \code{texts[text][counter]}
            \EndIf
            % \For{\code{i} $\gets 0$ to \code{k}}
            %     \State \code{word} $\gets$ \code{word} $+$ \code{texts[text][counter+i] $+$} `` ''
            % \EndFor
            % \State \code{word} $\gets$ strip right space from \code{word}
            \State{\textbf{try:}}
                \State{\hspace{6mm}\code{vals[word]} $\gets$ \code{vals[word]} $+ 1$}
            \State{\textbf{catch KeyNotFoundError:}}
                \Comment{Create new Key in Hashmap}
                \State{\hspace{6mm}\code{vals[word]} $\gets 1$}
            \State{\code{counter} $\gets$ \code{counter} $+1$}
        \EndWhile
        \State{\code{entries} $\gets$ \code{entries} $+$ \code{counter}}
    \EndFor
\algstore{countvectorizer}
\end{algorithmic}
\end{algorithm} %for page break
\begin{algorithm}[H]
\begin{algorithmic}[1]
\algrestore{countvectorizer}
    \State{\code{y} $\gets$ convert keys of \code{vals} to list}
    \State{\code{c} $\gets$ convert values of \code{vals} to list}
    \State{\code{arrlen\_} $\gets$ length of \code{y}}
    \If{\code{arrlen\_} = \code{entries}}
        \State{\textbf{return} \code{y,c}}
    \EndIf
    \State{\code{return\_vals} $\gets$ \code{MERGE\_SORT}(\code{y},\code{c},\code{0},\code{arrlen\_}$-1$})
    \State{\textbf{return} \code{return\_vals[0], return\_vals[1]}}
    \Comment{return\_vals is a (y,c) tuple}
    \EndFunction
    \end{algorithmic}
\end{algorithm}

\section{Laufzeitanalyse}
\label{sec:runtime}
Aus der Vorlesung ist bereits bekannt, dass \emph{Mergesort} eine asymptotische Laufzeit von $T(n) = \mathcal{O}(n\log(n))$ und einen Speicherverbrauch von $S(n) = \mathcal{O}(n)$ aufweist. 
\code{merge\_sort} und \code{merge\_} sortieren dabei die beiden Listen \code{words} und \code{weights} was zu mehreren Zuweisungen und zum Doppeltem Speicherverbrauch - im Vergleich zum \emph{Mergesort} mit einer Liste - führt. Da jedoch Konstanten nichts an der Ordnung der Laufzeit und des Speicherverbrauchs ändern bleibt dieser ordnungsmäßig gleich. 

Die gesamte Laufzeit und der Speicherverbrauch von \code{count\_vectorizer} ergibt sich somit wie folgt: Speicher und Laufzeit bis Zeile 9 ist $\mathcal{O}(1)$, da es sich nur um Initialisierung von Variablen handelt. Die folgende For-Schleife iteriert über die Listen mit den Wörtern aus den einzelnen Reviews und hat damit $T(m) = \mathcal{O}(m)$ wobei $m$ die Anzahl an Reviews angibt. Nun werden $len\_ - k$ ($len\_$ ... Wörter in der Review, $k$ Anzahl der aufeinanderfolgenden Wörter pro Phrase) Phrasen erstellt und in ein Dictionary gespeichert. $len\_-k$ wird im folgendem als $l$ bezeichnet und beschreibt die Länge der zu sortierenden Listen.
Diese While-Schleife braucht somit $\mathcal{O}(l)$ * $\mathcal{O}(n)$ (vom konkatenieren der Wörter $n$ ist Länge der ausgegebenen Phrase) Zeit und $\mathcal{O}(l)$ Speicher im Dictionary. 
Nach der For-Schleife werden die keys und values aus dem Dictionary in die Listen \code{y} und \code{c} gespeichert, dies benötigt lineare Zeit. Die weiteren Zeilen benötigen konstante Zeit - außer der \emph{Mergesort} welcher $T(n) = \mathcal{O}(n\log(n))$ benötigt - und fallen somit nicht weiter ins Gewicht. 

Somit ergibt sich die gesamte Zeitkomplexität zu: $T(l,m,n) = \mathcal{O}(m) * (\mathcal{O}(l) * \mathcal{O}(n)) + \mathcal{O}(l\log(l))$
Diese ergibt sich weiter zu $T(l) = \mathcal{O}(l) + \mathcal{O}(l\log(l)) = \mathcal{O}(l\log(l))$, da für jeden gegebenen Aufruf, $m$, eine Konstanten - je nach Anzahl der Reviews - und $n$ ebenfalls konstant, indem man die durchschnittliche Länge der Phrasen für $n$ verwendet, ist.


Der Speicherverbrauch ist durch $S(l) =\mathcal{O}(l)$ gegeben.



\section{Korrektheit des Algorithmus}
\label{sec:correctness}
Der Algorithmus durchläuft am Anfang alle Reviews und speichert sie in ein Dictionary, dies garantiert bereits, dass sich keine Duplikate unter den Phrasen von k-aufeinanderfolgenden Worten finden. Weiter kann man dadurch auch gleich die Auftrittshäufigkeit jeder Phrase in den Reviews bestimmen. Das Dictionary enthält somit alle einzigartigen Phrasen mit ihrer Häufigkeit und es werden daraus die beiden noch unsortierten Listen \code{y} und \code{c} generiert. Kommt jede Phrase genau einmal vor, so muss die Liste zwangsläufig sortiert sein (alle Listeneinträge in \code{c} sind schließlich 1) und der Algorithmus bricht vor dem sortieren ab. Sind die Listen nicht schon zufälligerweise sortiert, so werden diese mit \code{merge\_sort} sortiert. Hier ruft sich \code{merge\_sort} rekursiv auf und teilt die Liste dabei in eine linke und rechte Subliste bis schließlich einzelne Elemente erreicht werden, welche zurückgegeben werden. \code{merge\_} garantiert nun die richtige Sortierung dieser. Die linke und rechte Liste werden dabei verglichen, wobei das Element mit dem größeren weight immer zuerst gestellt wird. Das einfügen eines Tupels mit einer leeren Phrase und einem weight von 0 garantiert, dass nachdem eine der beiden Listen durchlaufen ist, immer ein Element aus der noch nicht vollständig durchlaufenen Liste zur Ausgabe hinzugefügt wird. Für den Fall, dass die Elemente in der linken und rechten Liste gleiches weight besitzten wird das linke bevorzugt, wodurch die ursprüngliche Reihenfolge inerhalb der Reviews (für gleichen count/weight) erhalten bleiben. 


\clearpage
% % Literaturverzeichnis
% \printbibliography

% % Abbildungsverzeichnis
% \listoffigures

% % Tabellenverzeichnis
% \listoftables

\end{document}
