% class
\documentclass[ngerman]{scrartcl}


% text input and font
\usepackage[utf8]{inputenc}  % input in UTF-8
\usepackage[T1]{fontenc}  % output in T1 fonts (west European encoding)
\usepackage{lmodern}  % Latin modern font
\usepackage{babel}  % language package
\usepackage{amsmath, amssymb, amstext, mathtools}  % math packages (American Math Society) + correction of amsmath (mathtools) [loads amsmath automatically]
\usepackage{physics}  % macros for easier typesetting of physical formulas
\usepackage{letltxmacro}  % \let command for robust macros (new sqrt)


% page geometry
\usepackage{scrlayer-scrpage}  % page formatting with KOMA options
\usepackage[paper=a4paper, hmargin=3cm, vmargin=2.5cm, includehead, includefoot]{geometry}


% floats
\usepackage[hypcap=false, labelfont=bf]{caption, subcaption}  % caption editing - hypcap warning with hyperref
\usepackage{float}  % for [H] (forced here) specifier
\usepackage{tabularray}
\usepackage{diagbox}  % table cells with diagonal lines


% graphical input
\usepackage{graphicx}  % input JPEG, PNG, PDF, etc.
\usepackage{pdfpages}  % input PDF as whole pages
\usepackage{lastpage}  % reference to last page


% text
\usepackage[locale=DE, uncertainty-mode=separate]{siunitx}  % SI units, German formatting - \pm stays \pm instead of ..(.)
\usepackage{icomma}  % no space after commas instead of English points) in decimal values
\usepackage{enumitem}  % better enumerating with style options
\usepackage{nicefrac}  % inline-fractions in n/d-style
\usepackage{fancyvrb}  % Verbatim environment with better options (capital V!)

% literacy
\usepackage[style=apa]{biblatex}  % backend=Biber is standard
\usepackage{csquotes}  % better quotation - should also be used in combination with package babel (warning)
\usepackage{xurl}  % breaks links - after BibLaTeX, but before hyperref!
\usepackage[hidelinks]{hyperref}  % produces most errors, last to load


% KOMA setups
% header and footer
\pagestyle{scrheadings}  % KOMA style
\clearpairofpagestyles  % reset
\setkomafont{pageheadfoot}{\normalfont}  % standard font in header and footer
\setlength{\headheight}{27.2pt}  % just look at the warning
\ihead{DA I\\Sortieralgorithmen}  % inner (left) head
\chead{\textsc{Wachmann} Elias (12004232)}  % center head
\ohead{03.11.2021}  % outer (right) head
\cfoot{\pagemark{} / \pageref*{LastPage}}  % center foot - *: ref but no hyperlink
% {}: empty statement
% \ : protected space
% \,: small space
\DeclareTOCStyleEntry{dottedtocline}{section}  % sections in TableOfContents with dotted lines
\KOMAoptions{parskip=half-}  % paragraphs with half a line height space instead of indentation, last line with no special treatment


% package setups

% BibLaTeX source
\addbibresource{halbleiterdiode.bib}


% rewrite names (babel overwrites German with standard English names, therefore at document beginn [after everything is loaded])
\AtBeginDocument{\renewcommand{\refname}{Literaturverzeichnis}}
% others:
% \contentsname
% \listtablename
% \listfigurename


% new sqrt
% https://en.wikibooks.org/wiki/LaTeX/Mathematics
\makeatletter
\let\oldr@@t\r@@t
\def\r@@t#1#2{%
\setbox0=\hbox{$\oldr@@t#1{#2\,}$}\dimen0=\ht0
\advance\dimen0-0.2\ht0
\setbox2=\hbox{\vrule height\ht0 depth -\dimen0}%
{\box0\lower0.4pt\box2}}
\LetLtxMacro{\oldsqrt}{\sqrt}
\renewcommand*{\sqrt}[2][\ ]{\oldsqrt[#1]{#2} }
\makeatother

 
% tabularray
% imports_and_setups{
%     expl3,
%     xparse,
%     ninecolors
%     \hypersetup{pdfborder={0 0 0}}
% }
\input{../tabularray-environments.tex}

\usepackage{pythonhighlight}
\usepackage{amsmath}
\usepackage{algorithm}
\usepackage[noend]{algpseudocode}

% individual settings
%\addbibresource{beschreibung.bib}  % database import with absolute path - file ending!
\newcommand{\code}{\texttt}



\begin{document}
\begin{titlepage}
    \begin{center}
        \vspace{1cm}
        \Huge
        \textbf{Data Structures and Algorithms I}
        \vspace{5mm}

        \Large
        Homework Assignment 3
        \vspace{5mm}


        \textbf{Wachmann Elias}


        \today
    \end{center}
\end{titlepage}

\clearpage
\tableofcontents
\newpage

\section{Algotithmus in Worten \& Pseudo-Code}
\label{sec:pseudo_code}
Aufgabenstellung ist es, aus einer Liste mit beliebig vielen Produkt-Reviews eine abwärts-sortierte Liste mit der Häufigkeit von k-hintereinander-stehenden Worten \code{c} und eine zugehörige Liste dieser Worte \code{y} zu generieren. 

\textbf{Der Algorithmus in Worten}

Dem Algorithmus \code{count\_vectorizer} wird eine list of lists namens \code{texts} und die Anzahl der aufeinanderfolgenden Wörter \code{k} übergeben.
Zuerst werden für jede Review in \code{texts} immer k-hintereinander-stehende Wörter in einem String konkateniert und in ein Python \code{Dict} gespeichert, hierbei wird zuerst versucht den String als Key zu verwenden, um so den count um $1$ zu erhöhen. Schlägt dies fehl, so wird ein neuer Key mit count = $1$ angelegt. Nun werden die Keys und Values des Dictionaries in die beiden Listen \code{y} bzw. \code{c} entpackt. Nun wird \code{c} mittels \code{merge\_sort} absteigend sortiert, dabei wird die Liste der Worte \code{y} gleich sortiert, sodass weiterhin jede Stelle in \code{c} die Anzahl der dazugehörigen Phrase in \code{y} gibt.

\textbf{Pseudo-Code}

\begin{algorithm}
\caption{\code{merge\_}}
\begin{algorithmic}[1]
    \Function{merge\_}{words, weights, start, k, end}
    
    \EndFunction
\end{algorithmic}
\end{algorithm}

\begin{algorithm}
\caption{\code{count\_vectorizer}}
\begin{algorithmic}[1]
    \Function{count\_vectorizer}{texts, k=1}
    \State{// Input: \code{texts} array of arrays with each containing the words from a review}
    \State{//        \code{k} integer with the count of consecutive words in a phrase (defaults to 1)}
    \State{// Output: \code{y} array of unique phrases (with \code{k} consecutive words)} 
    \State{// \code{c} array of occurrence count for corresponding index in \code{y}}
    \State $\code{y} \gets \text{[ ]}$
    \State $\code{c} \gets \text{[ ]}$
    \State \code{vals} $\gets$ \{ \}
    \State \code{entries} $\gets 0$
    \Comment{empty Hashmap}
    \For{\code{text} $\gets 0$ to $\text{length of } \code{texts}$}
        \State{\code{counter} $\gets 0$}
        \State{\code{len\_} $\gets \text{length of } \code{texts}$}
        \While{\code{counter} $\leq$ (\code{len\_} - \code{k})}
            % \State{\code{word} $\gets$ ""}
            \For{\code{i} $\gets 0$ to \code{k}}
                \State \code{word} $\gets$ \code{word} $+$ \code{texts[text][counter+i] $+$} `` ''
            \EndFor
            \State \code{word} $\gets$ strip right space from \code{word}
            \State{\textbf{try:}}
                \State{\hspace{6mm}\code{vals[word]} $\gets$ \code{vals[word]} $+ 1$}
            \State{\textbf{catch KeyNotFoundError:}}
                \Comment{Create new Key in Hashmap}
                \State{\hspace{6mm}\code{vals[word]} $\gets 1$}
            \State{\code{counter} $\gets$ \code{counter} $+1$}
        \EndWhile
        \State{\code{entries} $\gets$ \code{entries} $+$ \code{counter}}
    \EndFor
    \State{\code{y} $\gets$ convert keys of \code{vals} to list}
    \State{\code{c} $\gets$ convert values of \code{vals} to list}
    \State{\code{arrlen\_} $\gets$ length of \code{y}}
    \If{\code{arrlen\_} = \code{entries}}
        \State{\textbf{return} \code{y,c}}
    \EndIf
    \State{\code{return\_vals} $\gets$ \code{MERGE\_SORT}(\code{y},\code{c},\code{0},\code{arrlen\_}$-1$})
    \State{\textbf{return} \code{return\_vals[0], return\_vals[1]}}
    \Comment{return\_vals is a (y,c) tuple}
    \EndFunction
    \end{algorithmic}
\end{algorithm}



\clearpage
% Literaturverzeichnis
\printbibliography

% Abbildungsverzeichnis
\listoffigures

% Tabellenverzeichnis
\listoftables

\end{document}
